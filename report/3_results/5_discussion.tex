\section{Conclusions \& Future Work}
Throughout this project, we adapted cutting-edge techniques for the study of Cuprates to analyze
new optical conductivity data and did a thorough comparison with the older, more basic techniques.
However, due to the lack of data and the low quality of the existing data, we are unable to draw
strong conclusions.

The optical conductivity data from Legros et al. covers a very small range of frequencies, which
makes it difficult to extract the more intricate details of the shape of the data. The big errors
and noise in some of the data points also make it difficult to find a fit that best describes the
physics of the system.

Despite all this, we believe our results contrast that of Legros et al. enough to put the data
analysis into question. Our data also follows the general trend from Michon et al. more closely,
even if it is not definitive due to the large quantitative errors and the sparsity of the data.

Future work should focus on improving the quality of the data. Increasing the range of frequencies
for measuring the optical conductivity would make huge improvements. Reducing the noise and
systematic errors in the data would also be beneficial. Finally, more data points from more
doping levels allows for a better understanding of the trends in the data. Though there is already
data available for more doping levels, which we could not use because of a lack of availablity, and
the fact that most of it is in a doping range where our analysis breaks down and a more complex
model is needed. The explanation is beyond the scope of this report, but briefly, it is because the
Fermi surface and conductive properties of Cuprates are very different in the lower doping levels,
and it is not very well understood. It is still a subject of active research.