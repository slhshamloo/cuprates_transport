\subsection{Reproduction of the Drude fits}
We could reproduce most of the results of Legros et al. (Figure \ref{fig:drude_fit_good}).
However, no matter how much we fine-tuned the fitting procedure, we could not produce fits looking
as good as theirs for some of the plots (Figure \ref{fig:drude_fit_bad}). So, we tried fitting the
real part and the imaginary part of the optical conductivity separately, and we could produce
similar-looking fits. Extracting the parameters from these fits and comparing them to the parameters
extracted from the fits of Legros et al., we found that the parameters were consistent. This leads
us to believe this is indeed what Legros et al. did in their analysis. This is not ``correct'' in
some sense, but since the imaginary part of the optical conductivity is dominated by a
superconducting singularity, fitting it separately is a reasonable choice.

\begin{figure}
    \centering
    \includegraphics[width=0.7\textwidth]{figures/drude_fit_good.pdf}
    \caption{A good fit while fitting the real and imaginary parts of the optical conductivity
        simultaneously.}
    \label{fig:drude_fit_good}
\end{figure}
\begin{figure}
    \centering
    \includegraphics[width=0.7\textwidth]{figures/drude_fit_bad.pdf}
    \caption{A bad fit while fitting the real and imaginary parts of the optical conductivity
        simultaneously.}
    \label{fig:drude_fit_bad}
\end{figure}

%%% We can show the reproduction of Post et al. too maybe ?

\begin{figure}
    \centering
    \includegraphics[width=0.7\textwidth]{figures/fitting_Drude_extracted_Gamma.pdf}
    \caption{ADD SUBFIGURE A comparison of our and their fits with their data}
    \label{fig:drude_fit_good}
\end{figure}
