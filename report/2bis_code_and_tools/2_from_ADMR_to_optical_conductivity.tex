\subsection{From ADMR to optical conductivity}
% Our need to adapt it
With our model chosen and the experimental data acquired, 
we next needed to implement Chambers' formula to be able to fit the data and extract relevant parameters. 
To do that, we were able to leverage existing code and packages without reinventing the wheel.

% Context : There was already an existing code implementing Boltzmann transport calculations
Indeed, when studying cuprate high temperature superconductors, 
Prof. Grissonnanche's group extensively uses a code he developed during his earlier work on Angle Dependent Magnetoresistance (ADMR) experiments. 
This code is essentially implementing Chambers' formula in the DC case and is designed to vary the angle of the external field. 
Therefore, we could build upon this very mature codebase to extend it to any value of $\omega$ by modifying the right functions.

Our implementation introduces $\omega$ as an optional parameter defaulting to $0$. 
This means that it is a non-breaking change that doesn't affect past functionality, 
while allowing anyone to do optical conductivity calculations from now on.