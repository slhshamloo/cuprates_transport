\subsection{How to resolve the contradiction?}
In light of these contradicting conclusions, our project seeks to investigate a possible solution.
On the one hand, Michon et al.\cite{michon2019} probe criticality directly: 
that is, they look for a thermodynamic phase transition using a heat capacity measurement.

On the other hand, Legros et al.\cite{legros2022} probe it indirectly using a measurement of optical conductivity: 
$m_c$ is a parameter resulting from a fit of the experimental data. 
Thus, any results are dependent on the theoretical model chosen to fit the data.

An alternative to the Drude model used by Legros et al.\cite{legros2022} is the Chambers formula, 
derived from Boltzmann transport theory.  While the Drude model is derived for a
free electron (meaning it has isotropic transport properties), 
Boltzmann transport takes into account such anisotropy and more of the microscopic details of the
material.  This is particularly interesting for the study of cuprate high temperature
superconductors, where anisotropy is suspected to be of paramount importance.

