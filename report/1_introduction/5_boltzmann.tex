In light of these contradicting conclusions, our project seeks to investigate a possible solution.
On the one hand, \textit{Michon et al.} probe criticality directly : 
that is, they look for a thermodynamic phase transition using a heat capacity measurement.

On the other hand, \textit{Legros et al.} probe it indirectly using a measurement of optical conductivity : 
the value of interest is a parameter resulting from a fit of the experimental data. 
Thus, any results are dependent on the theoretical model chosen to fit the data.

One alternative to the Drude model used by \textit{Legros et al.} is Chambers' formula, 
derived from Boltzmann transport theory. 
While the Drude model is derived for a free electron 
(meaning it has isotropic transport properties), 
Boltzmann transport takes into account such anisotropy. 
This is particularly interesting for the study of cuprate high-Tc superconductors, 
where anisotropy is suspected to be of paramount importance.

