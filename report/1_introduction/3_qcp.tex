\subsection{Origin of Cuprate Superconductivity:\\The Quantum Critical Point}
The origin of high-temperature superconductivity in cuprates is still the subject of intense
research and debate, but a possible explanation has to do with a hypothetical Quantum Critical Point (QCP). 
In the phase diagram of a material, a QCP is defined as a point where there is a phase transition at zero temperature. 
Being at zero temperature (no thermal excitations), this phase transition is driven by quantum fluctuations, hence the name. 
This phase transition could lead to interactions that generate superconductivity. 
In the case of cuprates, this is a critical doping level, usually denoted by $p_c$ or $p^*$. 
Whether a QCP really exists in cuprates is still a matter of debate and is the subject of this project.

In their 2019 paper, Michon et al.\cite{michon2019} used specific heat measurements to
show the existence of the QCP. The effective mass of the charge carriers is expected to diverge at
the QCP, and the effective mass is proportional to $C/T$ where $C$ is the specific heat attributed
to the charge carriers and $T$ is the temperature. So, to probe the QCP, Michon et al.\cite{michon2019} measured
the specific heat at very low temperatures (1 to 2 K) and extrapolated to zero temperature. They
showed that the specific heat increases dramatically and peaks at the critical doping level
(Figure \ref{fig:michon}), which is a signature of a QCP.

The reason this divergence is expected is that the specific heat is related
to entropy, which is maximal at a thermodynamic phase transition. 
This is because the phase transition is a maximally disordered state, due to loss of scale (i.e. large correlation lengths) and high
fluctuations.\footnote{Again, this is related to loss of scale, but a deeper discussion of phase
transitions is beyond the scope of this report.}