\section{Mathematical Equivalence of Drude and Chambers for Free Electrons}
\label{sec:equivalence}
For free electrons, the first order path is a helix with $\phi(s) = \phi(0) - \omega_c s$
where $\phi$ is the azimuthal angle. Also, we are considering a free electron dispersion, it is
isotropic in $\vb{k}$-space:
\begin{itemize}
	\item We have the energy dispersion: $E(k) = \ddfrac{\hbar^2k^2}{2m^*}$
	\item The Fermi surface is a sphere of radius $k_F$ with $\vb{k} = k_F\vu{e}_r$
	\item On the FS, $\vb{v}_F = \ddfrac{1}{\hbar}\ddfrac{\partial E(\vb{k})}{\partial\vb{k}} = \ddfrac{1}{\hbar}\ddfrac{\hbar^2}{2m^*}2k_F\vu{e}_r = \ddfrac{\hbar}{m^*}k_F\vu{e}_r$
\end{itemize} 
Now, we apply the Chambers formula:
\begin{align}
	\sigma_{xx}^{\text{Chambers}} &= \frac{2e^2}{\hbar(2\pi)^3}\displaystyle\int_{\vb{k}}^{FS}\mathop{d\vb{k}}\int_0^{\infty}\mathop{ds}
		\hat{v}_x(\vb{k})v_x(\vb{k}(s))
		\exp{-i\omega s - \int_0^s \frac{\mathop{ds'}}{\tau \vb{k}(s')}} \\
	&= \frac{e^2}{4\hbar\pi^3}\displaystyle\int_{\vb{k}}^{FS}\mathop{d\vb{k}}\int_0^{\infty}\mathop{ds}
	\hat{v}_x(\vb{k})v_x(\vb{k}(s))
		e^{(i\omega - \Gamma)s} \\
	&= \frac{e^2}{4\hbar\pi^3}\displaystyle\int_{\vb{k}, k = k_F}^{}\mathop{d\vb{k}} \hat{v}_x(\vb{k}) 
		\int_0^{\infty}\mathop{ds} v_x(\vb{k}(s)) e^{(i\omega - \Gamma)s} \\
	&= \begin{aligned}[t] \frac{e^2}{4\hbar\pi^3}&\left(\displaystyle\int_{\phi = 0}^{2\pi}\int_{\theta = 0}^{\pi}\ k_F^2\sin{\theta}\mathop{d\phi}\mathop{d\theta}(\vu{e}_r(\theta, \phi)\cdot \vu{e}_x)\right. \\
		&\left.\int_0^{\infty}\mathop{ds} v_F (\vu{e}_r(\theta(s), \phi(s))\cdot \vu{e}_x) e^{(i\omega - \Gamma)s}\right) \end{aligned} \\
	&= \begin{aligned}[t] \frac{e^2}{4\hbar\pi^3}k_F^2v_F&\left(\displaystyle\int_{\phi = 0}^{2\pi}\int_{\theta = 0}^{\pi}\ \sin{\theta}\mathop{d\phi}\mathop{d\theta} \sin{\theta}\cos{\phi} \right. \\
		&\left.\int_0^{\infty}\mathop{ds} e^{(i\omega - \Gamma)s} \sin{\theta(s)}\cos{\phi(s)}\right) \end{aligned} \\
	&= \begin{aligned}[t] \frac{e^2}{4\hbar\pi^3}k_F^2v_F&\left(\displaystyle\int_{\phi = 0}^{2\pi}\int_{\theta = 0}^{\pi} \mathop{d\phi}\mathop{d\theta} \sin^3{\theta}\cos{\phi} \right. \\
		&\left.\int_0^{\infty}\mathop{ds} e^{(i\omega - \Gamma)s} \cos{(\phi - \omega_c s)}\right) \end{aligned} \\
	&\myeq\ \frac{e^2}{4\hbar\pi^3}k_F^2v_F\displaystyle\int_{\phi = 0}^{2\pi}\int_{\theta = 0}^{\pi} \mathop{d\phi}\mathop{d\theta} \sin^3{\theta}\frac{(\Gamma - i\omega)\cos^2{\phi}+\omega_c\cos\phi\sin\phi}{(\Gamma - i\omega)^2 + \omega_c^2} \\
	&= \frac{k_F^2v_Fe^2}{4\hbar\pi^3}\frac{\frac{4}{3}\pi(\Gamma - i\omega) + 0}{(\Gamma - i\omega)^2 + \omega_c^2} = \frac{k_F^2v_Fe^2}{3\hbar\pi^2}\frac{\Gamma - i\omega}{(\Gamma - i\omega)^2 + \omega_c^2}
\end{align}
Next, substituting $v_F$ in terms of $k_F$, we get
\begin{equation}
	\sigma_{xx}^{\text{Chambers}} = \frac{k_F^3e^2}{3\pi^2m^*}\frac{\Gamma - i\omega}{(\Gamma - i\omega)^2 + \omega_c^2}.
\end{equation}
Finally, $k_F = \qty(3\pi^2n)^{1/3}$ in 3D, so it gives
\begin{equation}
    \sigma_{xx}^{\text{Chambers}} = \frac{n e^2}{3m^*}\frac{\Gamma - i\omega}{(\Gamma - i\omega)^2 + \omega_c^2}.
\end{equation}

The derivation for the off-diagonal component is similar:
\begin{align}
	\sigma_{xy}^{\text{Chambers}} &= \frac{e^2}{4\pi^3}\displaystyle\int_{\vb{k}, k = k_F}^{}\mathop{d\vb{k}} \hat{v}_x(\vb{k}) 
		\int_0^{\infty}\mathop{ds} v_y(\vb{k}(s)) e^{(i\omega - \Gamma)s} \\
	&= \begin{aligned}[t] \frac{\hbar^2 e^2}{4\pi^3}&\left(\displaystyle\int_{\phi = 0}^{2\pi}\int_{\theta = 0}^{\pi}\ k_F^2\sin{\theta}\mathop{d\phi}\mathop{d\theta}(\vu{e}_r(\theta, \phi)\cdot \vu{e}_x)\right. \\
		&\left.\int_0^{\infty}\mathop{ds} v_F (\vu{e}_r(\theta(s), \phi(s))\cdot \vu{e}_y) e^{(i\omega - \Gamma)s}\right) \end{aligned} \\
	&= \begin{aligned}[t] \frac{\hbar^2 e^2}{4\pi^3}k_F^2v_F&\left(\displaystyle\int_{\phi = 0}^{2\pi}\int_{\theta = 0}^{\pi}\ \sin{\theta}\mathop{d\phi}\mathop{d\theta} \sin{\theta}\cos{\phi} \right. \\
		&\left.\int_0^{\infty}\mathop{ds} e^{(i\omega - \Gamma)s} \sin{\theta(s)}\sin{\phi(s)}\right) \end{aligned} \\
	&= \begin{aligned}[t] \frac{\hbar^2 e^2}{4\pi^3}k_F^2v_F&\left(\displaystyle\int_{\phi = 0}^{2\pi}\int_{\theta = 0}^{\pi} \mathop{d\phi}\mathop{d\theta} \sin^3{\theta}\cos{\phi} \right. \\
		&\left.\int_0^{\infty}\mathop{ds} e^{(i\omega - \Gamma)s} \sin{(\phi - \omega_c s)}\right) \end{aligned} \\
	&\myeq\ \frac{\hbar^2 e^2}{4\pi^3}k_F^2v_F\displaystyle\int_{\phi = 0}^{2\pi}\int_{\theta = 0}^{\pi} \mathop{d\phi}\mathop{d\theta} \sin^3{\theta}\frac{(\Gamma - i\omega)\cos{\phi}\sin\phi-\omega_c\cos^2\phi}{(\Gamma - i\omega)^2 + \omega_c^2} \\
	&= \frac{k_F^2v_Fe^2}{4\hbar\pi^3}\frac{0 - \frac{4}{3}\pi\omega_c}{(\Gamma - i\omega)^2 + \omega_c^2} = -\frac{\hbar^2 k_F^2v_Fe^2}{3\pi^2}\frac{\omega_c}{(\Gamma - i\omega)^2 + \omega_c^2}
\end{align}
\begin{equation}
	\sigma_{xy}^{\text{Chambers}} = -\frac{k_F^3e^2}{3\pi^2m^*}\frac{\omega_c}{(\Gamma - i\omega)^2 + \omega_c^2}
    = -\frac{n e^2}{3m^*}\frac{\omega_c}{(\Gamma - i\omega)^2 + \omega_c^2}
\end{equation}

After all these calculations, to convert to the circular basis for comparing to the Drude
model,\footnote{we rederived the Drude model in Cartesian basis too, it is just simpler to compare
to the circular basis already stated in the report and writing extra calculations for the Drude
model here is redundant.} we use the Kramers--Kronig relations.
\begin{align}
    \sigma_{ll/rr} &= \sigma_{xx} + i\sigma_{xy} = \frac{n e^2}{3m^*}
    \frac{\Gamma - i(\omega + \omega_c)}{(\Gamma - i\omega)^2 + \omega_c^2} \\
    &= \frac{n e^2}{3m^*}\frac{(\Gamma - i\omega) - i\omega_c}
    {[(\Gamma - i\omega) + i\omega_c][(\Gamma - i\omega) - i\omega_c]} \\
    &= \frac{n e^2}{3m^*}\frac{1}{\Gamma - i(\omega - \omega_c)} = \frac{n e^2}{3m^*}\frac{i}{\omega - \omega_c + i\Gamma}
\end{align}
When defining $\omega^2_{\mathrm{p,n}} = \frac{ne^2}{\epsilon_0 m^*}$, This is exactly the same as
the Drude formula in the circular basis, without the effect of supercunductivity.
